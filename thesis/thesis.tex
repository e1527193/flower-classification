% Copyright (C) 2014-2020 by Thomas Auzinger <thomas@auzinger.name>

\documentclass[draft,final]{vutinfth} % Remove option 'final' to obtain debug information.

% Load packages to allow in- and output of non-ASCII characters.
\usepackage{lmodern}        % Use an extension of the original Computer Modern font to minimize the use of bitmapped letters.
\usepackage[T1]{fontenc}    % Determines font encoding of the output. Font packages have to be included before this line.
\usepackage[utf8]{inputenc} % Determines encoding of the input. All input files have to use UTF8 encoding.

% Extended LaTeX functionality is enables by including packages with \usepackage{...}.
\usepackage{amsmath}    % Extended typesetting of mathematical expression.
\usepackage{amssymb}    % Provides a multitude of mathematical symbols.
\usepackage{mathtools}  % Further extensions of mathematical typesetting.
\usepackage{microtype}  % Small-scale typographic enhancements.
\usepackage[inline]{enumitem} % User control over the layout of lists (itemize, enumerate, description).
\usepackage{multirow}   % Allows table elements to span several rows.
\usepackage{booktabs}   % Improves the typesettings of tables.
\usepackage{subcaption} % Allows the use of subfigures and enables their referencing.
\usepackage[ruled,linesnumbered,algochapter]{algorithm2e} % Enables the writing of pseudo code.
\usepackage[usenames,dvipsnames,table]{xcolor} % Allows the definition and use of colors. This package has to be included before tikz.
\usepackage{nag}       % Issues warnings when best practices in writing LaTeX documents are violated.
\usepackage{todonotes} % Provides tooltip-like todo notes.
\usepackage{hyperref}  % Enables cross linking in the electronic document version. This package has to be included second to last.
\usepackage[acronym,toc]{glossaries} % Enables the generation of glossaries and lists fo acronyms. This package has to be included last.

% Define convenience functions to use the author name and the thesis title in the PDF document properties.
\newcommand{\authorname}{Tobias Eidelpes} % The author name without titles.
\newcommand{\thesistitle}{Flower State Classification for Watering System} % The title of the thesis. The English version should be used, if it exists.

% Set PDF document properties
\hypersetup{
    pdfpagelayout   = TwoPageRight,           % How the document is shown in PDF viewers (optional).
    linkbordercolor = {Melon},                % The color of the borders of boxes around crosslinks (optional).
    pdfauthor       = {\authorname},          % The author's name in the document properties (optional).
    pdftitle        = {\thesistitle},         % The document's title in the document properties (optional).
    pdfsubject      = {Subject},              % The document's subject in the document properties (optional).
    pdfkeywords     = {Object Detection, Image Classification, Machine Learning, Embedded Programming} % The document's keywords in the document properties (optional).
}

\setpnumwidth{2.5em}        % Avoid overfull hboxes in the table of contents (see memoir manual).
\setsecnumdepth{subsection} % Enumerate subsections.

\nonzeroparskip             % Create space between paragraphs (optional).
\setlength{\parindent}{0pt} % Remove paragraph identation (optional).

\makeindex      % Use an optional index.
\makeglossaries % Use an optional glossary.
%\glstocfalse   % Remove the glossaries from the table of contents.

% Set persons with 4 arguments:
%  {title before name}{name}{title after name}{gender}
%  where both titles are optional (i.e. can be given as empty brackets {}).
\setauthor{}{\authorname}{BSc}{male}
\setadvisor{Ao.Univ.-Prof. Dr.}{Horst Eidenberger}{}{male}

\setregnumber{01527193}
\setdate{20}{02}{2023} % Set date with 3 arguments: {day}{month}{year}.
\settitle{\thesistitle}{Flower State Classification for Watering System} % Sets English and German version of the title (both can be English or German).

% Select the thesis type: bachelor / master / doctor / phd-school.
% Master:
\setthesis{master}
\setmasterdegree{dipl.} % dipl. / rer.nat. / rer.soc.oec. / master

% For bachelor and master:
\setcurriculum{Software Engineering \& Internet Computing}{Software Engineering \& Internet Computing} % Sets the English and German name of the curriculum.

\newacronym{xai}{XAI}{Explainable Artificial Intelligence}
\newacronym{lime}{LIME}{Local Interpretable Model Agnostic Explanation}
\newacronym{grad-cam}{Grad-CAM}{Gradient-weighted Class Activation Mapping}

\begin{document}

\frontmatter % Switches to roman numbering.
% The structure of the thesis has to conform to the guidelines at
%  https://informatics.tuwien.ac.at/study-services

\addtitlepage{naustrian} % German title page (not for dissertations at the PhD School).
\addtitlepage{english} % English title page.
\addstatementpage

\begin{danksagung*}
\todo{Ihr Text hier.}
\end{danksagung*}

\begin{acknowledgements*}
\todo{Enter your text here.}
\end{acknowledgements*}

\begin{kurzfassung}
\todo{Ihr Text hier.}
\end{kurzfassung}

\begin{abstract}
\todo{Enter your text here.}
\end{abstract}

% Select the language of the thesis, e.g., english or naustrian.
\selectlanguage{english}

% Add a table of contents (toc).
\tableofcontents % Starred version, i.e., \tableofcontents*, removes the self-entry.

% Switch to arabic numbering and start the enumeration of chapters in the table of content.
\mainmatter

% \chapter{Introduction}
% \todo{Enter your text here.}

\chapter{Evaluation}

The following sections contain a detailed evaluation of the model in
various scenarios. First, we present metrics from the training phases
of the constituent models. Second, we employ methods from the field of
\gls{xai} such as \gls{lime} and \gls{grad-cam} to get a better
understanding of the models' abstractions. Finally, we turn to the
models' aggregate performance on the test set and discuss whether the
initial goals set by the problem description have been met or not.

\section{Object Detection}
\label{sec:eval-yolo}

The object detection model was trained for 300 epochs and the weights
from the best-performing epoch were saved. The model's fitness for
each epoch is calculated as the weighted average of \textsf{mAP}@0.5
and \textsf{mAP}@0.5:0.95:

\begin{equation}
  \label{eq:fitness}
  f_{epoch} = 0.1 \cdot \mathsf{mAP}@0.5 + 0.9 \cdot \mathsf{mAP}@0.5\mathrm{:}0.95
\end{equation}

Figure~\ref{fig:fitness} shows the model's fitness over the training
period of 300 epochs. The gray vertical line indicates the maximum
fitness of 0.61 at epoch 133. The weights of that epoch were frozen to
be the final model parameters. Since the fitness metric assigns the
\textsf{mAP} at the higher range the overwhelming weight, the
\textsf{mAP}@0.5 starts to decrease after epoch 30, but the
\textsf{mAP}@0.5:0.95 picks up the slack until the maximum fitness at
epoch 133. This is an indication that the model achieves good
performance early on and continues to gain higher confidence values
until performance deteriorates due to overfitting.

\begin{figure}
  \centering
  \includegraphics{graphics/model_fitness.pdf}
  \caption[Model fitness per epoch.]{Model fitness for each epoch
    calculated as in equation~\ref{eq:fitness}.}
  \label{fig:fitness}
\end{figure}

Overall precision and recall per epoch are shown in
figure~\ref{fig:prec-rec}. The values indicate that neither precision
nor recall change materially during training. In fact, precision
starts to decrease from the beginning, while recall experiences a
barely noticeable increase. Taken together with the box and object
loss from figure~\ref{fig:box-obj-loss}, we speculate that the
pre-trained model already generalizes well to plant detection. Any
further training solely impacts the confidence of detection, but does
not lead to higher detection rates. This conclusion is supported by
the increasing \textsf{mAP}@0.5:0.95.

\begin{figure}
  \centering
  \includegraphics{graphics/precision_recall.pdf}
  \caption{Overall precision and recall during training for each epoch.}
  \label{fig:prec-rec}
\end{figure}

Further culprits for the flat precision and recall values may be found
in bad ground truth data. The labels from the Open Images
Dataset~\cite{kuznetsova2020} are sometimes not fine-grained
enough. Images which contain multiple individual—often
overlapping—plants are labeled with one large bounding box instead of
multiple smaller ones. The model recognizes the individual plants and
returns tighter bounding boxes even if that is not what is specified
in the ground truth. Therefore, it is prudent to limit the training
phase to relatively few epochs in order to not penalize the more
accurate detections of the model. The smaller bounding boxes make more
sense considering the fact that the cutout is passed to the classifier
in a later stage. Smaller bounding boxes help the classifier to only
focus on one plant at a time and to not get distracted by multiple
plants in potentially different stages of wilting.

The box loss
decreases slightly during training which indicates that the bounding
boxes become tighter around objects of interest. With increasing
training time, however, the object loss increases, indicating that
less and less plants are present in the predicted bounding boxes. It
is likely that overfitting is a cause for the increasing object loss
from epoch 40 onward. Since the best weights as measured by fitness
are found at epoch 133 and the object loss accelerates from that
point, epoch 133 is probably the right cutoff before overfitting
occurs.

\begin{figure}
  \centering
  \includegraphics{graphics/val_box_obj_loss.pdf}
  \caption[Box and object loss.]{Box and object
    loss{\protect\footnotemark} measured against the validation set.}
  \label{fig:box-obj-loss}
\end{figure}

\footnotetext{The class loss is omitted because there is only one
  class in the dataset and the loss is therefore always 0.}

\begin{center}
\end{center}

\backmatter

% Use an optional list of figures.
\listoffigures % Starred version, i.e., \listoffigures*, removes the toc entry.

% Use an optional list of tables.
\cleardoublepage % Start list of tables on the next empty right hand page.
\listoftables % Starred version, i.e., \listoftables*, removes the toc entry.

% Use an optional list of alogrithms.
\listofalgorithms
\addcontentsline{toc}{chapter}{List of Algorithms}

% Add an index.
\printindex

% Add a glossary.
\printglossaries

% Add a bibliography.
\bibliographystyle{alpha}
\bibliography{references}

\end{document}
%%% Local Variables:
%%% mode: latex
%%% TeX-master: t
%%% End:
